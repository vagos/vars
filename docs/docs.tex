\documentclass{article}

% Packages

\usepackage{color}
\usepackage{listings}
\usepackage[hidelinks]{hyperref}

% Configuration

\setlength{\unitlength}{1mm} % Set the length that numerical units are measured in
\setlength{\parindent}{0pt} % Stop paragraph indentation
\setlength{\parskip}{1em}
\definecolor{lightgray}{gray}{0.9}

% Macros

\let\src\texttt
\renewcommand{\dots}{\ \dotfill{}\ } % Fills in the right amount of dots
\newcommand{\shortcut}[2]{#1~\dotfill{}~\src{#2}\\} % Custom command for adding a shorcut
% \newcommand{\src}[2][bash]{\colorbox{lightgray}{\lstinline[language=#1] #2}}

\newcommand{\me}{Vagos Labrou}
\newcommand{\terminalemulator}{Alacritty}
\newcommand{\editor}{vim}
\newcommand{\browser}{qutebrowser}
\newcommand{\altbrowser}{Brave}
\newcommand{\videoplayer}{mpv}
\newcommand{\wm}{i3}

\newcommand{\shortcuttitle}[1]{\underline{#1}}

% Info

\author{Vagos Labrou}
\title{V.A.R.S Documentation}
\date{}

\begin{document}

\maketitle
\tableofcontents

\section{Introduction}

Welcome to VARS! The official install script created by \me. Once the installation 
is done, you will get a fully featured Arch Linux installation filled with handpicked 
utilities that all combine into a cohesive workflow.

This documentation gives some guidelines on how to use your system. Obviously, 
these are all simply reccomendations; you should modify anything and everything
as you see fit.

\textbf{WARNING: This documentation is incomplete. You can figure most shortcuts out 
by looking at the configuration files located in 
\src{\char`~/.dotfiles}}

\section{Main Overview}


\subsection{Basic Navigation}

The window manager is \wm. It offers great flexibility on how you 
can arrange your windows. 

For screenshot taking I have only included the ability to take 
a shot of the active window because I feel thats the only sensible use case.

\subsubsection{Shortcuts}

\begin{minipage}{\textwidth}

\shortcut{New Terminal Window}{Mod + Enter}
\shortcut{Move between windows}{Mod + Arrowkeys}
\shortcut{Launch Programs with dmenu}{Mod + d}
\shortcut{Move around Workspaces}{Mod + Numeric Keys}
\shortcut{Move Window to Workspace}{Mod + Shift + Numeric Keys}

\end{minipage}

\begin{minipage}{\textwidth}
    \shortcuttitle{Change the organisation of windows}
    \newline

    \shortcut{Stack Windows}{Mod + w}
    \shortcut{Tile Windows}{Mod + e}

    \shortcut{Split the Window vertically}{Mod + v}
    \shortcut{Split the Window horizontally}{Mod + h}
\end{minipage}

\begin{minipage}{\textwidth}
    \shortcuttitle{Extras}

    \shortcut{Take screenshot of current window}{Mod + PrintScr}

\end{minipage}

\subsection{Text Editing}

The text editor is \editor. Most of this setup actually 
revolves around the usage of \editor{} shortcuts. You should 
look into the resources on how to use it. 

I have tried to include as few plugins as possible, leaving
things like git, file management and terminal multiplexing to other programs. 

Also, not too many custom shortcuts have been added, only adding 
some simple enhancements to already existing vim capabilities.

Auto completion is done through CoC (Conquer of Completion) since I found that it has 
the best out of the box experience. 

\subsubsection{Shortcuts}

\shortcut{Close \editor}{:q}

\shortcut{Install/Update your plugins}{:PlugInstall}
\shortcut{Install a code-completion engine through CoC}{:CocInstall \textless plugin-name\textgreater}

\subsection{Web}

Web browsing is done using \browser. It's fairly light and can be used 
exclusively using the keyboard. 

\subsubsection{Shortcuts}

\shortcut{Start search}{o}
\shortcut{Watch video using \videoplayer}{,m}

In case you want a browsing experience that includes password-storing 
and access to chrome plugins you can still use \altbrowser.

\subsection{Terminal}

The terminal emulator used is \terminalemulator. 

\subsection{Organisation}

When the installation is finished you will have three (3)
folders in your home directory: \src{files, projects, downloads}.

\begin{itemize}
    \item \src{files}: In here, you can put your media, documents and other useful stuff.
    \item \src{projects}: Each of your projects should live here in it's own folder.
    \item \src{downloads}: A temporary folder for downloads.
\end{itemize}

Folder names should have a homogenous naming-scheme as to avoid confusion when 
searching/browsing. What I find best is to only use lower-case names.

\subsection{Documents}

For document creation, you can use the typesetting system \LaTeX. 
It might have a steep learning curve but after creating some templates, 
you will be able to produce high quality documents faster and cleaner 
than with other methods.

\subsection{Entertainment}
For music, mpd is used as the music server with ncmpcpp
as the client. 

For interfacing with other services like Youtube and Spotify;
mopidy can be used instead. Mopidy also includes the ability to setup a web front-end to your music server 
so that you can control your computer's music from any other device on your local network.

\subsection{Syncing}

For syncing between computers (or folders) the utility rsync is used. 
Some useful scripts are included in the \src{files/scripts} folder.

\subsection{Other}

% Use makefiles!
% Use cheat.sh
% Search engine
% Torrents


\section{Contact Info}

\begin{itemize}
    \item e-mail: vagoslabrou@gmail.com
    \item discord: Vagoz\#9458
\end{itemize}

% \section{Resources}

\end{document}
